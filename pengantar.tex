\documentclass[10pt,a4paper]{article}
\usepackage[latin1]{inputenc}
\usepackage{amsmath}
\usepackage{amsfonts}
\usepackage{amssymb}
\usepackage{graphicx}

\begin{document}
	\title{Analisis Numerik}
	\author{Novalio Daratha}
	\maketitle
	\section{Pengertian}
	
	Sebelum mempelajari sesuatu, ada baiknya seseorang memahami manfaat hal tersebut. Dalam kuliah ini, Anda akan mempelajari "Analisis Numerik". Analisis Numerik meliputi perancangan dan analisis algoritma yang digunakan untuk memecahkan masalah matematis yang terkait dengan sains dan rekayasa. Ingatlah bahwa analisis numerik hanya membahas bilangan kontinyu (bukan bilangan diskrit). Jadi, ia dapat dianggap sebagai lanjutan dari mata kuliah Kalkulus.
	
	
	Mayoritas masalah dalam teknik elektro tidak bisa dipecahkan secara simbolik. Coba Anda sebutkan dua contoh saja! Saat mempelajari Rangkaian Listrik, Anda tentunya sudah menemukan bahwa analisis simbolik hanya cocok untuk rangkaian listrik yang sederhana dan jumlah elemennya sedikit.
\end{document}